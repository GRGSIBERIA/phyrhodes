\section{はじめに}

本研究では,電気楽器のシミュレーションについて述べる.本章では研究の背景や目的,方法について述べる.

\subsection{研究の背景・動機}

電気楽器とは,振動から電気を生成して音を鳴らす楽器のことである.電子楽器との違いは,電気楽器は生成された電気そのものが音響となるが,電子楽器はデジタル制御を含む点が異なる.本研究が対象とするのは,振動をもとに電気的に音響を生成する楽器である.

振動から電気的に音響を生成する楽器として思い浮かぶのはエレクトリック・ギターである.磁性体のスチール弦を振動させることで電磁ピックアップに誘導起電力を発生させ,アンプで音を増幅させる仕組みである.エレクトリック・ギターだけでも歴史的に複雑であるため割愛する.

電気楽器の魅力は,振動を電気的に増幅させる点である.電気の発生源として代表的なのが,ソレノイドコイルである.磁石に導線を巻くことで,磁場に変化が生じると導線に電気が流れる仕組みである.ほかに,圧電素子を利用し,振動を圧電効果によって電圧に変換する仕組みもある.電気をアンプで増幅させれば,大型のホールでも大音量で響かせることができる.音量を調整すれば家庭内でも騒音を撒き散らすことなく演奏もできる.

もともと楽器は室内楽に限定して開発されていた.それがホールなどの大部屋で多人数が視聴するようになって楽器の求められる性能として音量が挙げられるようになった.音量を大きくすれば屋内騒音が問題となり,音量を小さくすればホールで響かない.電気楽器はこれらのトレードオフを電気とアンプで解決する.

電気楽器は魅力的な一方で,設計が難しい側面もある.振動と電気の両側面から見なければならないため,関わる分野が極めて多様である.例えば,振動体の構造は機械工学や材料工学,ソレノイドコイルを取り付ければ電磁気学が必須である.電子回路を搭載すれば電子工学も関わるようになる.電気楽器は極めて複雑性の高い分野と言える.また,ピアノやバイオリンなどの古典的な楽器も未だ未解明の現象が存在している.電気楽器は何をモデルとするかでアンプで増幅させる電気が変化する.バイオリンに圧電素子を取り付ければアンプで増幅できるようになるため,電気を作る前に音色をどのように設計するか検討しなければならない.


% 研究の背景・研究の動機
% 研究の目的
%% リサーチクエスチョンを立てる
% 研究の方法
% 用語の定義
% 論文の構成