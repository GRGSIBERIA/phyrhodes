\section{ローズ・ピアノの簡易物理モデル}

本章ではローズ・ピアノ振動体を簡略化した物理モデルを検討する.本章では,ローズ・ピアノのシステムモデルを連立常微分方程式として表し,TineとTonebarの連成振動を考慮した簡易物理モデルを提案する.

\subsection{簡易物理モデル}

簡易物理モデルに使用するシステムモデルを図\ref{fig:簡易モデル}に示す.

\begin{figure}
    \centering
    \includegraphics[width=15cm]{img/system-model.png}
    \caption{ローズ・ピアノの簡易物理モデル}
    \label{fig:簡易モデル}
\end{figure}

図はローズ・ピアノのシステムモデルである.$k$はバネ定数,$m$は質量,$u$は変位,$c$はダッシュポット,$l$は節の長さ,$\theta$は節の角度である.図中上部がTonebarになり,図中下部がTineに相当する.TonebarとTineをつないでいるPoleは$m_3$に相当する.

質点に着目すると,$m_1$がTine先端,$m_6$がTonebar先端である.$m_2$はTuning Springとなっている.$m_3$はHammerの打点であり,$m_7$はHammerである.Tonebarは3つに分けて考えている.$m_4$はPoleが取り付けられているところである.$m_5$はTonebarがねじれている部分であり,$m_6$はTonebarが共鳴器として作用する部分である.$m_7$はHammerであるため実際にはリンクしないが,Hammerと打点の衝突をモデル化するためにダッシュポット$c_7$が存在する.$m_7$は初期速度$v_0$で変位し,$m_3$と衝突する.$m_7$はダッシュポットによって速度比例の減衰力が働くので,$m_3$に急激な変位は発生しない.

Hammerを除き,境界条件を含めると質点は7つ存在する.本章ではラグランジュ方程式を利用するため角度を用いるが,各質点が回転対偶な自由度を持つとすると,リンクは6つ存在する.全体の自由度$N$は

\begin{eqnarray}
    N = 3(n - 1) - 2n_1-n_2
\end{eqnarray}

で表される.このとき$n$はリンクの数,$n_1$は自由度1の回転対偶の総数,$n_2$は自由度2の回転対偶の総数である.リンクの数及び自由度に代入すると,

\begin{eqnarray}
    \begin{matrix}
        N &=& 3\times (6 - 1) - 2 \times 4 - 1 \\
          &=& 15 - 8 - 1 = 6
    \end{matrix}
\end{eqnarray}

となるので,ラグランジアンで角度を対象とすると7質点6自由度のバネ-マス-ダンパー系の物理モデルとなる.並進を考慮すると,

\begin{eqnarray}
    \begin{matrix}
        N &=& 3\times (6 - 1) - 2 \times 0 - 6 \\
          &=& 15 - 0 - 6 = 9
    \end{matrix}
\end{eqnarray}

となるので,7質点9自由度となり,計15自由度のバネ-マス-ダンパー系の物理モデルとなる.


\subsection{ラグランジュ方程式}

簡易物理モデルは3ないしは4重振り子の複雑な物理モデルとなっている.変位を一般化する前に,ラグランジュ方程式で変位を表せることを検討する.

ラグランジュ方程式は

\begin{eqnarray}
    L = T - U    
\end{eqnarray}

で与えられる方程式である.$T$は運動エネルギ,$U$はポテンシャルエネルギ,$L$はラグランジアンである.

簡易物理モデルを$u$を$x$と$y$の座標成分として表すと

\begin{eqnarray}
    \begin{matrix}
        x_1 &=& l_1 \cos \theta_1 &+& l_2 \cos \theta_2 &+& l_3 \cos \theta_3 &+& l_4 \cos \theta_4\\
        x_2 &=& l_2 \cos \theta_2 &+& l_3 \cos \theta_3 &+& l_4 \cos \theta_4 & & \\
        x_3 &=& l_3 \cos \theta_3 &+& l_4 \cos \theta_4 & & & &\\
        x_4 &=& l_4 \cos \theta_4 & & & & & &\\
        x_5 &=& l_5 \cos \theta_5 &+& l_4 \cos \theta_4 & & & &\\
        x_6 &=& l_6 \cos \theta_6 &+& l_5 \cos \theta_5 &+& l_4 \cos \theta_4 & &\\
    \end{matrix}
\end{eqnarray}

\begin{eqnarray}
    \begin{matrix}
        y_1 &=& l_1 \sin \theta_1 &+& l_2 \sin \theta_2 &+& l_3 \sin \theta_3 &+& l_4 \sin \theta_4\\
        y_2 &=& l_2 \sin \theta_2 &+& l_3 \sin \theta_3 &+& l_4 \sin \theta_4 & & \\
        y_3 &=& l_3 \sin \theta_3 &+& l_4 \sin \theta_4 & & & &\\
        y_4 &=& l_4 \sin \theta_4 & & & & & &\\
        y_5 &=& l_5 \sin \theta_5 &+& l_4 \sin \theta_4 & & & &\\
        y_6 &=& l_6 \sin \theta_6 &+& l_5 \sin \theta_5 &+& l_4 \sin \theta_4 & & \\
    \end{matrix}
\end{eqnarray}

である.$u_7$はHammerであり,位置関係は拘束されていないので本節では除外する.

運動エネルギ$T$は

\begin{eqnarray}
    \label{math:lagrange-T}
    T &=& \sum_{i = 1}^{6} \frac{1}{2} m_i({x'}_i^2 + {y'}_i^2) \nonumber \\
      &=& \sum_{i = 1}^{6} T_i
\end{eqnarray}

である.このとき,$\theta$を時刻$t$の時間関数とすると合成関数の微分であるため,$(l\sin \theta)' = l \dot{\theta} \cos \theta$と$(l\cos \theta)' = -l \dot{\theta} \sin \theta$となる.

式\ref{math:lagrange-T}を展開すると,

\begin{eqnarray}
        T_1 &=& m_{1} \left(- \dot{\theta_1} l_{1} \sin{\left(\theta_1 \right)} - \dot{\theta_2} l_{2} \sin{\left(\theta_2 \right)} - \dot{\theta_3} l_{3} \sin{\left(\theta_3 \right)} - \dot{\theta_4} l_{4} \sin{\left(\theta_4 \right)}\right)^{2} \nonumber \\
            & & + m_{1} \left(\dot{\theta_1} l_{1}\cos{\left(\theta_1 \right)} + \dot{\theta_2} l_{2} \cos{\left(\theta_2 \right)} + \dot{\theta_3} l_{3} \cos{\left(\theta_3 \right)} + \dot{\theta_4} l_{4} \cos{\left(\theta_4 \right)}\right)^{2}
\end{eqnarray}
\begin{eqnarray}
        T_2 &=& m_{2} \left(- \dot{\theta_2} l_{2} \sin{\left(\theta_2 \right)} - \dot{\theta_3} l_{3} \sin{\left(\theta_3 \right)} - \dot{\theta_4} l_{4} \sin{\left(\theta_4 \right)}\right)^{2} \nonumber \\
            & & + m_{2} \left(\dot{\theta_2} l_{2} \cos{\left(\theta_2 \right)} + \dot{\theta_3} l_{3}\cos{\left(\theta_3 \right)} + \dot{\theta_4} l_{4} \cos{\left(\theta_4 \right)}\right)^{2} 
\end{eqnarray}
\begin{eqnarray}
        T_3 &=& m_{3} \left(- \dot{\theta_3} l_{3} \sin{\left(\theta_3 \right)} - \dot{\theta_4} l_{4} \sin{\left(\theta_4 \right)}\right)^{2} \nonumber \\
            & & + m_{3} \left(\dot{\theta_3} l_{3} \cos{\left(\theta_3 \right)} + \dot{\theta_4} l_{4} \cos{\left(\theta_4 \right)}\right)^{2}
\end{eqnarray}
\begin{eqnarray}
        T_4 &=& m_{4} \left(\dot{\theta_4} l_{4} \sin{\left(\theta_4 \right)} + \dot{\theta_4} l_{4} \cos{\left(\theta_4 \right)}\right)^{2}
\end{eqnarray}
\begin{eqnarray}
        T_5 &=& m_{5} \left(- \dot{\theta_4} l_{4} \sin{\left(\theta_4 \right)} - \dot{\theta_5} l_{5} \sin{\left(\theta_5 \right)}\right)^{2} \nonumber \\
            & & + m_{5} \left(\dot{\theta_4} l_{4} \cos{\left(\theta_4 \right)} + \dot{\theta_5} l_{5} \cos{\left(\theta_5 \right)}\right)^{2}
\end{eqnarray}
\begin{eqnarray}
        T_6 &=& m_{6} \left(- \dot{\theta_4} l_{4} \sin{\left(\theta_4 \right)} - \dot{\theta_5} l_{5} \sin{\left(\theta_5 \right)} - \dot{\theta_6} l_{6} \sin{\left(\theta_6 \right)}\right)^{2} \nonumber \\
            & & + m_{6} \left(\dot{\theta_4} l_{4} \cos{\left(\theta_4 \right)} + \dot{\theta_5} l_{5}\cos{\left(\theta_5 \right)} + \dot{\theta_6} l_{6} \cos{\left(\theta_6 \right)}\right)^{2}
\end{eqnarray}

与えられた式の形から$T$の形で整理できないため,運動エネルギは$T$とする.


\subsection{連立常微分方程式}

主変数 $u$ に関する連立常微分方程式は,

\begin{equation}
    M \frac{d^2 u}{dt^2} + B_c^T R B_c \frac{du}{dt} + B_k^T D B_k = f    
\end{equation}

である.$M$は質量マトリクス,$D$はバネマトリクス,$R$は減衰マトリクス,$B_c$と$B_k$は剛性マトリクス,$f$は荷重マトリクスである.連立常微分方程式を粘弾性の運動方程式で一般化すると

\begin{eqnarray}
    m\ddot{u} + c\dot{u} + ku = f
\end{eqnarray}

である.$u$は質点の変位,$m$は質点の質量,$c$は減衰,$k$はバネ定数である.微分形式を省略するため,特に明記がない限り2階の時間微分は$\ddot{u}$,1階の時間微分は$\dot{u}$で表す.

図\ref{fig:簡易モデル}より,運動方程式からなる連立方程式は

\begin{eqnarray}
    \begin{matrix}
        m_1 \ddot{u_1} &+&  & & k_1 (u_1 - u_2) &=& 0 \\ 
        m_2 \ddot{u_2} &+& c_2(\dot{u_2} - \dot{u_3}) &+& k_2 (u_2 - u_3) - k_1 (u_2 - u_1) &=& 0 \\ 
        m_3 \ddot{u_3} &+& c_2(\dot{u_3} - \dot{u_2}) - c_3(\dot{u_3} - \dot{u_4}) &+& k_2 (u_3 - u_2) - k_3 (u_3 - u_4) &=& 0 \\ 
        m_4 \ddot{u_4} &+& c_3(\dot{u_4} - \dot{u_3}) &+& k_3 (u_4 - u_3) - k_5 (u_4 - u_5) - k_4 u_4 &=& 0 \\ 
        m_5 \ddot{u_5} &+& c_6(\dot{u_5} - \dot{u_6}) &+& k_6 (u_5 - u_6) - k_5 (u_5 - u_4) &=& 0 \\
        m_6 \ddot{u_6} &+& c_6(\dot{u_6} - \dot{u_5}) &+& k_6 (u_6 - u_5) &=& 0 \\
    \end{matrix}        
\end{eqnarray}

である.連立方程式は変位に着目したとき,対象の変数と接続する変位との差を取ることで求まる.例えば,$u_2$の$c_2$及び$k_2$に着目するならば,$u_2$と接続している変位$u_3$について式を立てると,$c_2(\dot{u_2} - \dot{u_3})$と$k_2(u_2 - u_3)$が得られる.同じように$u_2$と接続している$u_1$に着目すると,$u_1$を接続しているのは$k_1$だけなので減衰項は$0$として$k_2(u_2 - u_1)$だけが得られる.それぞれの項を足し合わせると$u_2$に関する運動方程式を立てられる.これをすべての変位で繰り返すと連立方程式となる.

連立方程式より状態方程式は,

\begin{eqnarray}
    u = 
    \begin{bmatrix}
        u_1 \\
        u_2 \\
        u_3 \\
        u_4 \\
        u_5 \\
        u_6 
    \end{bmatrix}
\end{eqnarray}

\begin{eqnarray}
    f = 
    \begin{bmatrix}
        f_1 \\
        f_2 \\
        f_3 \\
        f_4 \\
        f_5 \\
        f_6 
    \end{bmatrix}
\end{eqnarray}

\begin{eqnarray}
    M = 
    \left(\begin{matrix}
        m_1 & 0   & 0   & 0   & 0   & 0   \\
        0   & m_2 & 0   & 0   & 0   & 0   \\
        0   & 0   & m_3 & 0   & 0   & 0   \\
        0   & 0   & 0   & m_4 & 0   & 0   \\
        0   & 0   & 0   & 0   & m_5 & 0   \\
        0   & 0   & 0   & 0   & 0   & m_6 
    \end{matrix}\right)
\end{eqnarray}

\begin{eqnarray}
    D =
    \left(\begin{matrix}
        k_1 & 0   & 0   & 0   & 0   & 0    \\
        0   & k_2 & 0   & 0   & 0   & 0    \\
        0   & 0   & k_3 & 0   & 0   & 0    \\
        0   & 0   & 0   & k_4 & 0   & 0    \\
        0   & 0   & 0   & 0   & k_5 & 0    \\
        0   & 0   & 0   & 0   & 0   & k_6  
    \end{matrix}\right)
\end{eqnarray}

\begin{eqnarray}
    R = 
    \left(\begin{matrix}
        0   & 0   & 0   & 0   & 0   & 0   \\
        0   & c_2 & 0   & 0   & 0   & 0   \\
        0   & 0   & c_3 & 0   & 0   & 0   \\
        0   & 0   & 0   & 0   & 0   & 0   \\
        0   & 0   & 0   & 0   & 0   & 0   \\
        0   & 0   & 0   & 0   & 0   & c_6 
    \end{matrix}\right)
\end{eqnarray}

である.

剛性マトリクスは連立方程式にかかる剛性マトリクスから行列を作成し,重ね合わせの原理によって足し合わせることで求められる.単に$k$もしくは$c$に対して何倍なのか求めればいいだけであるため,連立方程式より

\begin{eqnarray}
    B_k = 
    \left(\begin{matrix}
        1   & -1  & 0   & 0   & 0  & 0  \\
        -1  & 2   & -1  & 0   & 0  & 0  \\
        0   & -1  & 2   & -1  & 0  & 0  \\
        0   & 0   & -1  & 3   & -1 & 0  \\
        0   & 0   & 0   & -1  & 2  & -1 \\
        0   & 0   & 0   & 0   & -1 & 1  
    \end{matrix}\right)
\end{eqnarray}

\begin{eqnarray}
    B_c = 
    \left(\begin{matrix}
        0   & 0   & 0   & 0   & 0  & 0  \\
        0   & 1   & -1  & 0   & 0  & 0  \\
        0   & -1  & 2   & -1  & 0  & 0  \\
        0   & 0   & -1  & 1   & 0  & 0  \\
        0   & 0   & 0   & 0   & 1  & -1 \\
        0   & 0   & 0   & 0   & -1 & 1  
    \end{matrix}\right)
\end{eqnarray}

が得られる.よって,

\begin{eqnarray}
    M\ddot{u} =
    \left[\begin{matrix}
        m_1 \ddot{u_1} \\
        m_2 \ddot{u_2} \\
        m_3 \ddot{u_3} \\
        m_4 \ddot{u_4} \\
        m_5 \ddot{u_5} \\
        m_6 \ddot{u_6} 
    \end{matrix}\right]
\end{eqnarray}

\begin{eqnarray}
    B_k^T R B_k \dot{u} =
    \left[\begin{matrix}
        0 \\
        c_2 (\dot{u_2} - \dot{u_3}) + c_3 (\dot{u_3} - \dot{u_4}) \\
        c_3 (\dot{u_3} - \dot{u_4}) \\
        0 \\
        0 \\
        0 \\
        c_6 (\dot{u_6} - \dot{u_5}) 
    \end{matrix}\right]
\end{eqnarray}

\begin{eqnarray}
    B_k^T D B_k u =
    \left[\begin{matrix}
        k_1 (u_1 - u_2) \\
        k_2 (u_2 - u_3) - k_1 (u_1 - u_2) \\
        k_3 (u_3 - u_4) - k_2 (u_2 - u_1) - k_1 (u_1 - u_2) \\
        k_4 u_4 - k_3 (u_3 - u_2) - k_2 (u_2 - u_1) - k_1 (u_1 - u_2) - k_5 (u_4 - u_5) - k_6 (u_5 - u_6) \\
        k_5 (u_5 - u_4) - k_6 (u_6 - u_5) \\
        k_6 (u_6 - u_5) 
    \end{matrix}\right]
\end{eqnarray}

である.
