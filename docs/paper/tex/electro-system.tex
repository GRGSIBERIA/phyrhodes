\section{電磁誘導解析システム}

ローズ・ピアノ振動体についてシミュレーションを行った.シミュレーションによってローズ・ピアノ振動体がTineとTonebarの間で連成振動が発生していることを確認できた.ローズ・ピアノは電気楽器であり,電磁ピックアップによって誘導起電力を得て音響に変換している.本章では,振動体によって電磁ピックアップに生じる誘導起電力を算出するシステムについて述べる.

\subsection{開発背景}

電磁場解析(Computational Electro-Magnetics)ソリューションはFemapを始めとする様々なソフトウェアで提供されている.ソルバも様々な手法が開発されている.我々はダッソー・システムズのAbaqusを利用してシミュレーションを行っている.ダッソー・システムズが電磁場解析ソリューションを提供し始めたのは2016年にComputer Simulation Technologyを買収してからである.ダッソー・システムズは電磁場解析ソリューションを提供している企業を買収しながら,Abaqusで電磁場解析ができるように発展させている.本稿執筆時点(2019年11月1日)ではAbaqusにおいて誘導起電力を導出するソルバはAbaqus/Electromagneticsで提供されている.しかし,現状ではAbaqus/ElectromagneticsとAbaqus/Standardとソルバを併存させることができず,動解析から電磁場解析を行うことができない.

Abaqus/Standard & Explicitで静解析及び動解析を実行して結果を閲覧することは珍しくない.解析結果をもとにAbaqus/Electromagneticsを実行したいが,それぞれ別々のソリューションであるため相互に連携できない問題がある.本章で述べる電磁場解析システムは,Abaqus/Standard & Explicitで解析した結果をもとに電磁場解析を行うシステムを提案する.

\subsection{電磁場解析の着眼点}

電磁場解析には磁界の解析からヒステリシス損失,誘導起電力など,多様な分野での解析方法が存在する.本稿で述べる電磁場解析は電磁誘導解析に限定する.電磁誘導解析とは,電磁誘導によって引き起こされるヒステリシス損失や誘導起電力を解析するための手法である.電磁誘導とは,磁場が時刻によって変化すると,磁場に存在する導体に電位差が生じる現象のことである.代表的な仕組みとしてソレノイドコイルが挙げられる.ソレノイドコイルとは,螺旋もしくは渦巻状に導体を巻きつけた形状のことである.ソレノイドコイルの渦巻の内側に磁石を設置すると,導体が磁場に曝される.このとき,導体の磁場に変化が生じれば磁場の変化と直行するように電位差が生じる.電位差によって導体内に渦電流が生じるが導体も抵抗を持っているため熱としてエネルギーが損失する.本稿ではこのような渦電流損については考慮しないものとする.電磁誘導解析の目的の一つとして,磁場内に存在する導体に生じる電位差を解析することが挙げられる.


\subsubsection{ファラデーの電磁誘導の法則}
