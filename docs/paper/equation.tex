\section{ローズ・ピアノの物理モデル}

本章ではローズ・ピアノ振動体を簡略化した物理モデルを検討する.簡易モデルに使用するシステムモデルを図\figref{簡易モデル}に示す.

% ![ローズ・ピアノ簡易モデル](img/system-model.png)

図はローズ・ピアノのシステムモデルである.$k$はバネ定数,$m$は質量,$u$は変位,$c$はダッシュポットである.なお,$k_5$はねじりバネである.図中上部がTonebarになり,図中下部がTineに相当する.TonebarとTineをつないでいるPoleは$m_4$と$m_3$に相当する.

\subsection{連立常微分方程式}

主変数 $u$ に関する連立常微分方程式は,

\begin{equation}
    M \frac{d^2 u}{dt^2} + B^T R B \frac{du}{dt} + B^T D B_u = f    
\end{equation}

である.$M$は質量マトリクス,$D$はバネマトリクス,$R$は減衰マトリクス,$B$は固有ベクトルをまとめた対角行列マトリクスである.

\figref{簡易モデル}を運動方程式は,

\begin{eqnarray}
    m_1 \ddot{u_1} + c_1(\dot{u_1} - \dot{u_2}) + k_1 (u_1 - u_2) &=& 0 \\
    m_2 \ddot{u_2} + c_2(\dot{u_2} - \dot{u_1}) + k_2 (u_2 - u_3) + k_1 (u_2 - u_1) &=& 0 \\
    m_3 \ddot{u_3} + c_2(\dot{u_3 - \dot{u_2}}) + k_2 (u_3 - u_2) + k_3 u_3 &=& 0
\end{eqnarray}

である.運動方程式より状態方程式を起算すると,